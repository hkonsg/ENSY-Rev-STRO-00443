\section*{Acknowledgments}
We gratefully acknowledge support from the research center HydroCen, RCN No.\ 257588. We are thankful to Montel for supplying price data, and to Eviny for supplying data regarding their hydropower plants.



\section{Introduction}
Sequential decision-making in the presence of uncertainty is of relevance in many diverse applications, including inventory control problems, production scheduling, and system modelling \cite{bertsekas2012dynamic}. Typically, the goal is to minimize cost or maximize profits by dynamically adapting decisions to exogenous information. These problems are often computationally intractable because of exponential growth in the number of scenarios needed to evaluate future expected costs or profits. Therefore, approximate dynamic programming (ADP) algorithms are often used to obtain near-optimal policies \cite{powell2007a}.

In some applications, one can utilize the problem structure in developing computationally tractable algorithms, such as decomposition-based algorithms \cite{van1969shaped,infanger1996cut}. The stochastic dual dynamic programming algorithm (SDDP), is an example \cite{pereira1991a}. This algorithm is state of the art for solving the seasonal hydropower production planning problem \cite{gjelsvik2010a,shapiro2012}. The algorithm requires the problem formulation to be convex and uncertainty to be stage-wise independent \cite{philpott2008a}. Therefore, certain aspects of the real problem need to be approximated, to adhere to the convexity requirement. To handle non-convexities, some recent papers have proposed to extend the algorithm \cite{hjelmeland2018nonconvex,downward2020stochastic}, or to approximate the problem using McCormick envelopes \cite{cerisola2012a}. To adhere to the requirement of stage-wise independence, serial correlation is typically accounted for by either imposing a linear dependence structure and increasing the state vector, or discretizing the state space before applying the algorithm \cite{Lohndorf2019}. The first approach greatly restricts the dynamics of exogenous factors, while the latter approach induces a discretization error.

An alternative class of algorithms that often provide good policies is forecast-based reoptimization heuristics, e.g.\ the rolling intrinsic (RI) heuristic \cite{breslin2009gas}. The heuristic repeatedly solve deterministic problems based on the expectation of exogenous factors. When averaging revenues from sufficiently many sample paths, it provides a feasible policy and an estimated lower bound on maximization problems. Reoptimization heuristics are widely used by practitioners in various fields. In gas storage management, numerical experiments indicate that the rolling intrinsic algorithm provides near-optimal policies for price- and storage-dependent injections and withdrawals \cite{lai2010a,wu2012a}. Despite the ability to handle complex relationships, few works have assessed the performance of reoptimization-based heuristics in hydropower scheduling. An exception, albeit for short-term operations, is \cite{LoWa20}.

In this work, our main contributions are to provide an extensive comparison of the rolling intrinsic algorithm, the state-of-the-art algorithm SDDP, and a novel, heuristic-based, algorithm that does not impose any restrictions on the modelling of exogenous factors. We call this algorithm Scenario-based Two-stage ReOptimizaton (STRO). The main innovative aspect is that the requirement that the algorithm must yield a policy is relaxed. Still, the algorithm provides an estimate of the lower bound on the optimal value, from which marginal water values can be derived. This concept is referred to as \emph{policy relaxation}.

%explorations of Hmm - trodde vi ble enige om at vi skulle framstille dette som at scenario generation kunne være en tilnærming, her hopper vi rett på sampling.
STRO samples possible future outcomes and solves two-stage programs repeatedly to make decisions. The idea is that this will better capture the range of possible realizations of exogenous factors, compared to the rolling intrinsic heuristic which uses a deterministic forecast based on expectations. Capturing the range of possible realizations is crucial in applications where all risk factors cannot be hedged, e.g.\ a hydropower producer which may risk spillage. In the setting of hydropower scheduling using SDDP, our algorithm allows for a more realistic representation of inflow dynamics, e.g.\ by using dynamic artificial neural networks \cite{Valipour2012}, as no restrictions are imposed on the modelling of the exogenous factors. Another benefit of the algorithm is the opportunity for highly scalable parallel processing. Because scenarios are generated independently, their respective policies can be computed individually. The use of parallelization allows for a drastic reduction in the processing time required. As seen in previous works, traditional parallelization schemes do not scale \textcolor{red}{to the same extent} for SDDP \cite{AvPL21,helseth_braaten_2015}. Finally, the algorithm does not explicitly require the problem formulation to be convex, unlike SDDP. We provide an extensive numerical comparison of optimality gaps and computation time of the proposed algorithm, the state-of-the-art algorithm 
SDDP, and the rolling intrinsic algorithm. 

The most important findings are that both the RI and STRO methods achieve revenues that are less than 3\% from the estimate of the upper bound on optimal revenue based on several trial runs of the SDDP algorithm. Furthermore, by increasing the number of samples per decision for the STRO method, the performance of the algorithm increases substantially. When using 2 or more samples per decision, STRO outperforms RI by a decent margin and achieves revenues less than 2\% from the estimated upper bound. The benefit of increasing the number of samples for the STRO heuristic is largest when moving from 1 to 2 samples, and the benefit from increasing the number of samples further was found to diminish quickly.

The paper is structured as follows. First, a detailed mathematical formulation of the problem is given, along with an in-depth problem description, in section \ref{section:mathematical model}. Then the novel heuristic is presented in section \ref{subsection:proposed_method_generalization}. We then give an illustrative example of how this heuristic works in section \ref{section: illustrative example}. In section \ref{case and implementation} we give an overview of the specific case the algorithms were tested on, as well as the software and hardware used. Finally, the results of our experiments are outlined in section \ref{section:comp_study} before we give our final remarks in section \ref{Chapter concluding remarks}.