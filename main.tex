% This is samplepaper.tex, a sample chapter demonstrating the
% LLNCS macro package for Springer Computer Science proceedings;
% Version 2.20 of 2017/10/04
%
\documentclass[runningheads]{llncs}
%
\usepackage{graphicx}
\usepackage{import} 
\usepackage{amsmath}    
\usepackage{amssymb} 
\let\proof\relax 
\let\endproof\relax
\usepackage{amsthm}
\usepackage{multirow,booktabs}
\usepackage{rotating}
\usepackage{makecell}
\usepackage{float}
\usepackage{xcolor}
\usepackage{eurosym}
\usepackage{caption}
\usepackage{subcaption}


\usepackage{accents}
\newlength{\dhatheight}
\newcommand{\doublehat}[1]{%
    \settoheight{\dhatheight}{\ensuremath{\hat{#1}}}%
    \addtolength{\dhatheight}{-0.35ex}%
    \hat{\vphantom{\rule{1pt}{\dhatheight}}%
    \smash{\hat{#1}}}}

% Used for displaying a sample figure. If possible, figure files should
% be included in EPS format.
%
% If you use the hyperref package, please uncomment the following line
% to display URLs in blue roman font according to Springer's eBook style:
% \renewcommand\UrlFont{\color{blue}\rmfamily}

\begin{document}
%
\title{A policy relaxation algorithm for seasonal hydropower planning}
%
\titlerunning{ADP for hydropower reservoir management}
% If the paper title is too long for the running head, you can set
% an abbreviated paper title here
%
\author{Håkon S. Grini\inst{1} \and
Anders S. Danielsen\inst{1} \and
Stein-Erik Fleten\inst{1} \and
Andreas Kleiven\inst{1}}
%
\authorrunning{H. Grini et al.}
% First names are abbreviated in the running head.\orcidID{0000-0001-6462-0266} is for Fleten
% If there are more than two authors, 'et al.' is used.
%
\institute{Norwegian University of Science and Technology, 7491 Trondheim, Norway}
%
\maketitle              % typeset the header of the contribution
%
\begin{abstract}
Hydropower producers need to plan several months or years ahead to estimate the opportunity value of water stored in their reservoirs. The resulting large-scale optimization problem is computationally intensive, and model simplifications are often needed to allow for efficient solving. Alternatively, one can look for near-optimal policies using heuristics that can tackle non-convexities in the production function and %general models for the uncertain environment. 
a wide range of modelling approaches for the price- and inflow dynamics. 
We undertake an extensive numerical comparison between the state-of-the-art algorithm stochastic dual dynamic programming (SDDP) and rolling forecast-based algorithms, including a novel algorithm that we develop in this paper. We name it Scenario-based Two-stage ReOptimization abbreviated as STRO.  The numerical experiments are based on convex stochastic dynamic programs with discretized exogenous state space, which makes the SDDP algorithm applicable for comparisons. We demonstrate that our algorithm can handle inflow risk better than traditional forecast-based algorithms, by reducing the optimality gap from 2.5\% to 1.3\% compared to the SDDP bound.

\keywords{Stochastic programming \and Policy relaxation \and Hydropower reservoir management.}
\end{abstract}
%
%
%
\import{}{1_intro}
\import{}{2_model}
\import{}{3_proposed_method}
\import{}{4_case}
\import{}{5_comp_study}
\import{}{6_concluding_remarks}






%
% ---- Bibliography ----
%
% BibTeX users should specify bibliography style 'splncs04'.
% References will then be sorted and formatted in the correct style.
%
\bibliography{references}{}
\bibliographystyle{splncs04}
\appendix
\import{}{model_appendix}
% \bibliography{mybibliography}
%
\end{document}
